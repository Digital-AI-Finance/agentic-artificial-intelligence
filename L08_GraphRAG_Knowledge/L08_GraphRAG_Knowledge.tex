\documentclass[8pt,aspectratio=169]{beamer}
\usetheme{Madrid}
\usepackage{graphicx}
\usepackage{booktabs}
\usepackage{amsmath}
\usepackage{amssymb}
\usepackage{hyperref}

\definecolor{mlblue}{RGB}{0,102,204}
\definecolor{mlpurple}{RGB}{51,51,178}
\definecolor{mllavender}{RGB}{173,173,224}
\definecolor{mllavender2}{RGB}{193,193,232}
\definecolor{mllavender3}{RGB}{204,204,235}
\definecolor{mllavender4}{RGB}{214,214,239}
\definecolor{mlorange}{RGB}{255,127,14}
\definecolor{mlgreen}{RGB}{44,160,44}
\definecolor{mlgray}{RGB}{127,127,127}

\setbeamercolor{palette primary}{bg=mllavender3,fg=mlpurple}
\setbeamercolor{palette secondary}{bg=mllavender2,fg=mlpurple}
\setbeamercolor{palette tertiary}{bg=mllavender,fg=white}
\setbeamercolor{palette quaternary}{bg=mlpurple,fg=white}
\setbeamercolor{structure}{fg=mlpurple}
\setbeamercolor{frametitle}{fg=mlpurple,bg=mllavender3}

\setbeamertemplate{navigation symbols}{}
\setbeamertemplate{itemize items}[circle]
\setbeamersize{text margin left=5mm,text margin right=5mm}

\newcommand{\bottomnote}[1]{\vfill\vspace{-2mm}\textcolor{mllavender2}{\rule{\textwidth}{0.4pt}}\vspace{1mm}\footnotesize\textbf{#1}}

\title{GraphRAG and Knowledge Integration}
\subtitle{Week 8: From Vector Search to Knowledge Graphs}
\author{Agentic Artificial Intelligence}
\institute{PhD Course}
\date{2025}

\begin{document}

\setbeamertemplate{footline}{\hbox{\begin{beamercolorbox}[wd=\paperwidth,ht=2.5ex,dp=1ex,center]{author in head/foot}\tiny (c) Joerg Osterrieder 2025\end{beamercolorbox}}}

\begin{frame}[plain]
\vspace{1.5cm}
\begin{center}
{\Huge\textcolor{mlpurple}{GraphRAG and Knowledge Integration}}\\[0.5cm]
{\Large Week 8: From Vector Search to Knowledge Graphs}\\[1.5cm]
{\normalsize PhD Course in Agentic Artificial Intelligence}
\end{center}
\end{frame}

\begin{frame}[t]{Learning Objectives}
\textbf{Bloom's Taxonomy Levels}
\begin{itemize}
\item \textbf{Remember}: Define knowledge graphs, entities, relations, communities
\item \textbf{Understand}: Explain how GraphRAG enhances retrieval with structure
\item \textbf{Apply}: Build a knowledge graph from unstructured text
\item \textbf{Analyze}: Compare vector-only vs graph-enhanced retrieval
\item \textbf{Evaluate}: Assess when GraphRAG provides value over standard RAG
\item \textbf{Create}: Design a hybrid retrieval system for a domain
\end{itemize}
\bottomnote{By end of lecture, you will understand structured knowledge integration in agents.}
\end{frame}

\begin{frame}[t]{GraphRAG Architecture}
\begin{center}
\includegraphics[width=0.60\textwidth]{01_graphrag_architecture/graphrag_architecture.pdf}
\end{center}
\bottomnote{GraphRAG builds structure from documents before retrieval.}
\end{frame}

\begin{frame}[t]{Entity and Relationship Extraction}
\begin{center}
\includegraphics[width=0.60\textwidth]{02_entity_extraction/entity_extraction.pdf}
\end{center}
\bottomnote{LLMs extract entities and relations to build the knowledge graph.}
\end{frame}

\begin{frame}[t]{Community Detection}
\begin{center}
\includegraphics[width=0.60\textwidth]{03_community_detection/community_detection.pdf}
\end{center}
\bottomnote{Leiden algorithm clusters related entities for hierarchical summarization.}
\end{frame}

\begin{frame}[t]{Query Routing by Type}
\begin{center}
\includegraphics[width=0.60\textwidth]{04_query_routing/query_routing.pdf}
\end{center}
\bottomnote{Different query types benefit from different retrieval strategies.}
\end{frame}

\begin{frame}[t]{Key Papers}
\textbf{This Week}
\begin{itemize}
\item Edge et al. (2024). ``From Local to Global: A GraphRAG Approach.'' Microsoft Research
\item Pan et al. (2024). ``Unifying Large Language Models and Knowledge Graphs.'' arXiv:2306.08302
\end{itemize}
\vspace{0.3cm}
\textbf{Supplementary}
\begin{itemize}
\item Besta et al. (2024). ``Graph of Thoughts.'' arXiv:2308.09687
\item Gutierrez et al. (2024). ``HippoRAG.'' arXiv:2405.14831
\end{itemize}
\bottomnote{Focus on the Microsoft GraphRAG paper for implementation details.}
\end{frame}

\begin{frame}[t]{Summary}
\textbf{Key Concepts}
\begin{itemize}
\item \textbf{GraphRAG}: Combine knowledge graphs with vector retrieval
\item \textbf{Entity Extraction}: LLM-based NER and relation extraction
\item \textbf{Communities}: Hierarchical clustering for global queries
\item \textbf{Hybrid Retrieval}: Route queries to appropriate strategy
\end{itemize}
\vspace{0.3cm}
\textbf{Next Week}
\begin{itemize}
\item Hallucination Prevention and Verification
\end{itemize}
\bottomnote{GraphRAG = Structure + Vectors for comprehensive retrieval.}
\end{frame}

\end{document}
