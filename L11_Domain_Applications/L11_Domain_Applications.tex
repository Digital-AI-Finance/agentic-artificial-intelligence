\documentclass[8pt,aspectratio=169]{beamer}
\usetheme{Madrid}
\usepackage{graphicx}
\usepackage{booktabs}
\usepackage{adjustbox}
\usepackage{amsmath}
\usepackage{amssymb}
\usepackage{hyperref}

% Color definitions
\definecolor{mlblue}{RGB}{0,102,204}
\definecolor{mlpurple}{RGB}{51,51,178}
\definecolor{mllavender}{RGB}{173,173,224}
\definecolor{mllavender2}{RGB}{193,193,232}
\definecolor{mllavender3}{RGB}{204,204,235}
\definecolor{mllavender4}{RGB}{214,214,239}
\definecolor{mlorange}{RGB}{255,127,14}
\definecolor{mlgreen}{RGB}{44,160,44}
\definecolor{mlred}{RGB}{214,39,40}
\definecolor{mlgray}{RGB}{127,127,127}

\setbeamercolor{palette primary}{bg=mllavender3,fg=mlpurple}
\setbeamercolor{palette secondary}{bg=mllavender2,fg=mlpurple}
\setbeamercolor{palette tertiary}{bg=mllavender,fg=white}
\setbeamercolor{palette quaternary}{bg=mlpurple,fg=white}
\setbeamercolor{structure}{fg=mlpurple}
\setbeamercolor{frametitle}{fg=mlpurple,bg=mllavender3}
\setbeamercolor{block title}{bg=mllavender2,fg=mlpurple}
\setbeamercolor{block body}{bg=mllavender4,fg=black}

\setbeamertemplate{navigation symbols}{}
\setbeamertemplate{itemize items}[circle]
\setbeamersize{text margin left=5mm,text margin right=5mm}

\newcommand{\bottomnote}[1]{%
\vfill
\vspace{-2mm}
\textcolor{mllavender2}{\rule{\textwidth}{0.4pt}}
\vspace{1mm}
\footnotesize\textbf{#1}
}

\title{Domain Applications}
\subtitle{Week 11: Code and Finance Agents}
\author{Agentic Artificial Intelligence}
\institute{PhD Course}
\date{2025}

\begin{document}

\setbeamertemplate{footline}{
    \hbox{\begin{beamercolorbox}[wd=\paperwidth,ht=2.5ex,dp=1ex,center]{author in head/foot}
    \tiny (c) Joerg Osterrieder 2025
    \end{beamercolorbox}}
}

% ==================== SLIDE 1: Title ====================
\begin{frame}[plain]
\vspace{1.5cm}
\begin{center}
{\Huge\textcolor{mlpurple}{Domain Applications}}\\[0.5cm]
{\Large Week 11: Code, Finance, and Healthcare Agents}\\[1.5cm]
{\normalsize PhD Course in Agentic Artificial Intelligence}\\[0.5cm]
{\small 12-Week Research-Level Course}
\end{center}
\end{frame}

% ==================== SLIDE 2: Learning Objectives ====================
\begin{frame}[t]{Learning Objectives}
\textbf{Bloom's Taxonomy Levels Covered}
\begin{itemize}
\item \textbf{Remember}: Define SWE-bench, code agent, FinAgent, regulatory compliance
\item \textbf{Understand}: Explain domain-specific requirements for agent deployment
\item \textbf{Apply}: Implement a code agent using flow engineering (structured pipelines)
\item \textbf{Analyze}: Compare agent architectures across different domains
\item \textbf{Evaluate}: Assess regulatory and safety requirements for each domain
\item \textbf{Create}: Design a domain-specific agent with appropriate safeguards
\end{itemize}
\bottomnote{By end of lecture, you will understand how agents adapt to real-world domains.}
\end{frame}

% ==================== SLIDE 3: Domain Maturity Landscape ====================
\begin{frame}[t]{Domain Maturity Landscape}
\textbf{High Maturity: Software Development}
\begin{itemize}
\item Clear success criteria (tests pass, code works)
\item Sandboxed execution environments
\item Active deployment: GitHub Copilot, Cursor, Devin
\end{itemize}
\vspace{0.3cm}
\textbf{Medium-High Maturity: Finance}
\begin{itemize}
\item Well-defined tasks (analysis, research, reporting)
\item Heavy regulatory constraints (SEC, FINRA, MiFID II)
\item Active deployment: Trading assistants, document analysis, compliance
\end{itemize}
\vspace{0.3cm}
\textbf{Finance Sub-Domains}
\begin{itemize}
\item \textbf{Research}: High maturity (summarization, analysis)
\item \textbf{Trading}: Medium maturity (backtesting safe, live trading risky)
\item \textbf{Compliance}: Growing (document review, audit trails)
\end{itemize}
\bottomnote{Maturity correlates with ability to verify outputs and contain errors.}
\end{frame}

% ==================== SLIDE 4: Application Domain Maturity Chart ====================
\begin{frame}[t]{Application Domain Maturity}
\begin{center}
\includegraphics[width=0.60\textwidth]{01_application_domains/application_domains.pdf}
\end{center}
\bottomnote{Software development leads in maturity; healthcare is emerging.}
\end{frame}

% ==================== SLIDE 5: Code Agents ====================
\begin{frame}[t]{Code Agents: The Leading Domain}
\textbf{Why Code is Ideal for Agents}
\begin{itemize}
\item Clear success criteria: Tests pass or fail
\item Safe sandbox: Run code in containers
\item Immediate feedback: Execution reveals errors
\item Rich context: Codebase provides grounding
\end{itemize}
\vspace{0.3cm}
\textbf{Key Capabilities}
\begin{itemize}
\item Bug fixing and debugging
\item Feature implementation from specifications
\item Code review and refactoring
\item Documentation generation
\end{itemize}
\vspace{0.3cm}
\textbf{Current State}
\begin{itemize}
\item SWE-bench: Best agents solve $\sim$50\% of real GitHub issues
\item Production systems: Copilot, Cursor, Devin, Claude Code
\end{itemize}
\bottomnote{Code agents now outperform average developers on specific benchmarks.}
\end{frame}

% ==================== SLIDE 6: SWE-bench and AlphaCodium ====================
\begin{frame}[t]{SWE-bench and AlphaCodium}
\textbf{SWE-bench (Jimenez et al., 2024)}
\begin{itemize}
\item 2,294 real GitHub issues from 12 Python repositories
\item Task: Generate code patch to resolve issue
\item Verification: Patch must pass repository tests
\end{itemize}
\vspace{0.3cm}
\textbf{AlphaCodium: Flow Engineering (Ridnik et al., 2024)}
\begin{itemize}
\item Structured multi-stage pipeline (not single-shot)
\item Stages: Problem reflection, public tests, AI tests, code iteration
\item Key insight: Test against multiple cases before submitting
\end{itemize}
\vspace{0.3cm}
\textbf{Flow Engineering Principles}
\begin{itemize}
\item Break complex tasks into simpler stages
\item Generate and run tests iteratively
\item Use structured output at each stage
\end{itemize}
\bottomnote{Flow engineering = structured pipelines for complex coding tasks.}
\end{frame}

% ==================== SLIDE 7: Code Agents Chart ====================
\begin{frame}[t]{Code Agents: SWE-bench Performance}
\begin{center}
\includegraphics[width=0.60\textwidth]{02_code_agents/code_agents.pdf}
\end{center}
\bottomnote{Code agents now outperform average human developers on SWE-bench.}
\end{frame}

% ==================== SLIDE 8: Finance Agents ====================
\begin{frame}[t]{Finance Agents}
\textbf{High-Value Applications}
\begin{itemize}
\item \textbf{Research}: Earnings analysis, market research synthesis
\item \textbf{Trading}: Strategy backtesting, execution assistance
\item \textbf{Compliance}: Regulatory document analysis, audit trails
\item \textbf{Operations}: Report generation, data reconciliation
\end{itemize}
\vspace{0.3cm}
\textbf{Unique Challenges}
\begin{itemize}
\item \textbf{Regulatory}: SEC, FINRA, MiFID II compliance requirements
\item \textbf{Explainability}: Must justify recommendations
\item \textbf{Latency}: Markets move in milliseconds
\item \textbf{Risk}: Errors have direct financial consequences
\end{itemize}
\vspace{0.3cm}
\textbf{Current Deployments}
\begin{itemize}
\item FinAgent: Multimodal trading agent (research)
\item Bloomberg Terminal AI: Document analysis, Q\&A
\end{itemize}
\bottomnote{Finance requires compliance (regulatory) awareness at every step.}
\end{frame}

% ==================== SLIDE 9: Finance Agents Chart ====================
\begin{frame}[t]{Finance Agent Applications}
\begin{center}
\includegraphics[width=0.65\textwidth]{03_finance_agents/finance_agents.pdf}
\end{center}
\bottomnote{Finance agents span research, trading, compliance, and operations.}
\end{frame}

% ==================== SLIDE 10: FinAgent Architecture ====================
\begin{frame}[t]{FinAgent: Multimodal Finance Agent}
\textbf{Architecture (Li et al., 2024)}
\begin{itemize}
\item Multimodal: Text (news, filings), numeric (prices, fundamentals), charts
\item Dual memory: Short-term (recent trades), long-term (market patterns)
\item Tool use: Market data APIs, technical indicators, portfolio analytics
\end{itemize}
\vspace{0.3cm}
\textbf{Key Components}
\begin{itemize}
\item \textbf{Market Perception}: Process multi-modal market signals
\item \textbf{Agent Memory}: Store and retrieve trading experience
\item \textbf{Decision Module}: ReAct-style reasoning for trade decisions
\end{itemize}
\vspace{0.3cm}
\textbf{Results}
\begin{itemize}
\item Outperforms baselines on paper trading benchmarks
\item Caveat: Simulated environment, not live trading
\end{itemize}
\bottomnote{Multimodal perception is critical for financial markets.}
\end{frame}

% ==================== SLIDE 11: Finance Agent Architectures ====================
\begin{frame}[t]{Finance Agent Architectures}
\textbf{Research Agents}
\begin{itemize}
\item Earnings call analysis and summarization
\item SEC filing extraction (10-K, 10-Q, 8-K)
\item News sentiment aggregation across sources
\end{itemize}
\vspace{0.3cm}
\textbf{Trading Agents}
\begin{itemize}
\item Strategy backtesting with historical data
\item Signal generation from technical/fundamental indicators
\item Portfolio rebalancing recommendations
\end{itemize}
\vspace{0.3cm}
\textbf{Compliance Agents}
\begin{itemize}
\item Regulatory document parsing (MiFID II, Dodd-Frank)
\item Trade surveillance and anomaly detection
\item Audit trail generation and reporting
\end{itemize}
\bottomnote{Different finance tasks require different agent architectures.}
\end{frame}

% ==================== SLIDE 12: Finance Regulatory Compliance ====================
\begin{frame}[t]{Regulatory Compliance in Finance Agents}
\textbf{Key Regulations}
\begin{itemize}
\item \textbf{SEC/FINRA} (US): Suitability rules, best execution, record-keeping
\item \textbf{MiFID II} (EU): Transparency, investor protection, reporting
\item \textbf{Basel III}: Capital requirements, risk management
\end{itemize}
\vspace{0.3cm}
\textbf{Agent Compliance Patterns}
\begin{itemize}
\item \textbf{Audit logging}: Every decision must be traceable
\item \textbf{Explainability}: Justify recommendations to regulators
\item \textbf{Human oversight}: Compliance officer approval for actions
\item \textbf{Data governance}: Handle PII and market data appropriately
\end{itemize}
\vspace{0.3cm}
\textbf{Risk}: Unexplainable AI decisions = regulatory violations
\bottomnote{Compliance-by-design is mandatory for production finance agents.}
\end{frame}

% ==================== SLIDE 13: Risk Management in Trading Agents ====================
\begin{frame}[t]{Risk Management in Trading Agents}
\textbf{Risk Categories}
\begin{itemize}
\item \textbf{Market risk}: Position limits, VaR constraints, stop-losses
\item \textbf{Execution risk}: Slippage, failed orders, latency
\item \textbf{Model risk}: Strategy drift, overfitting, regime change
\end{itemize}
\vspace{0.3cm}
\textbf{Agent Safeguards}
\begin{itemize}
\item Hard position limits (cannot be overridden by agent)
\item Kill switches for automated trading
\item Human approval above threshold sizes
\item Real-time P\&L monitoring with alerts
\end{itemize}
\vspace{0.3cm}
\textbf{Key Principle}: Agents recommend, humans execute high-risk trades
\bottomnote{Risk controls must be enforced at infrastructure level, not by the agent.}
\end{frame}

% ==================== SLIDE 14: Finance Agent Case Studies ====================
\begin{frame}[t]{Finance Agent Case Studies}
\textbf{Bloomberg Terminal AI}
\begin{itemize}
\item Document Q\&A over financial filings
\item Earnings call summarization
\item Human-in-loop for all outputs
\end{itemize}
\vspace{0.3cm}
\textbf{Quantitative Research Assistants}
\begin{itemize}
\item Alpha factor discovery from alternative data
\item Automated literature review for trading ideas
\item Strategy prototyping (not live execution)
\end{itemize}
\vspace{0.3cm}
\textbf{Compliance Automation}
\begin{itemize}
\item KYC document verification
\item Transaction monitoring for AML
\item Regulatory report generation
\end{itemize}
\bottomnote{Current focus: Research and compliance; trading execution remains human-controlled.}
\end{frame}

% ==================== SLIDE 15: Cross-Domain Patterns ====================
\begin{frame}[t]{Cross-Domain Design Patterns}
\textbf{Verification Strategy by Domain}
\begin{itemize}
\item \textbf{Code}: Run tests, syntax checking, type checking
\item \textbf{Finance Research}: Cross-reference sources, fact-check numbers
\item \textbf{Finance Trading}: Backtesting, risk limits, compliance rules
\end{itemize}
\vspace{0.3cm}
\textbf{Human-in-the-Loop Intensity}
\begin{itemize}
\item \textbf{Code}: Low (automated tests catch most errors)
\item \textbf{Finance Research}: Medium (analyst review of summaries)
\item \textbf{Finance Trading}: High (human execution for significant trades)
\end{itemize}
\vspace{0.3cm}
\textbf{Common Success Factors}
\begin{itemize}
\item Domain-specific tools and knowledge bases
\item Clear escalation paths for uncertainty
\item Audit trails for accountability
\end{itemize}
\bottomnote{Adapt verification intensity to domain risk level.}
\end{frame}

% ==================== SLIDE 16: Key Papers ====================
\begin{frame}[t]{Required Readings}
\textbf{This Week}
\begin{itemize}
\item Jimenez et al. (2024). ``SWE-bench: Can Language Models Resolve Real-World GitHub Issues?'' arXiv:2310.06770
\item Ridnik et al. (2024). ``AlphaCodium: Code Generation with Flow Engineering.'' arXiv:2401.08500
\item Li et al. (2024). ``FinAgent: A Multimodal Foundation Agent for Financial Trading.'' arXiv:2402.18485
\end{itemize}
\vspace{0.3cm}
\textbf{Supplementary}
\begin{itemize}
\item Yang et al. (2024). ``SWE-agent: Agent-Computer Interfaces Enable Software Engineering.'' arXiv:2405.15793
\item Lopez-Lira \& Tang (2023). ``Can ChatGPT Forecast Stock Price Movements?'' arXiv:2304.07619
\end{itemize}
\bottomnote{Focus on SWE-bench for code agents and FinAgent for finance agents.}
\end{frame}

% ==================== SLIDE 17: Summary ====================
\begin{frame}[t]{Summary and Key Takeaways}
\textbf{Domain Insights}
\begin{itemize}
\item \textbf{Code}: Most mature; clear success criteria, safe sandboxing
\item \textbf{Finance Research}: High value; summarization and analysis
\item \textbf{Finance Trading}: High risk; requires strict safeguards
\item \textbf{Finance Compliance}: Growing rapidly; audit and documentation
\end{itemize}
\vspace{0.3cm}
\textbf{Design Principles}
\begin{itemize}
\item Match verification intensity to domain risk
\item Build domain-specific tools and knowledge
\item Design clear human escalation paths
\end{itemize}
\vspace{0.3cm}
\textbf{Next Week}
\begin{itemize}
\item Research Frontiers and Final Projects
\end{itemize}
\bottomnote{Domain expertise + agent capabilities = real-world impact.}
\end{frame}

\end{document}
