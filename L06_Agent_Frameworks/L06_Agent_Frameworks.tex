\documentclass[8pt,aspectratio=169]{beamer}
\usetheme{Madrid}
\usepackage{graphicx}
\usepackage{booktabs}
\usepackage{adjustbox}
\usepackage{amsmath}
\usepackage{amssymb}
\usepackage{hyperref}

% Color definitions
\definecolor{mlblue}{RGB}{0,102,204}
\definecolor{mlpurple}{RGB}{51,51,178}
\definecolor{mllavender}{RGB}{173,173,224}
\definecolor{mllavender2}{RGB}{193,193,232}
\definecolor{mllavender3}{RGB}{204,204,235}
\definecolor{mllavender4}{RGB}{214,214,239}
\definecolor{mlorange}{RGB}{255,127,14}
\definecolor{mlgreen}{RGB}{44,160,44}
\definecolor{mlred}{RGB}{214,39,40}
\definecolor{mlgray}{RGB}{127,127,127}

% Apply custom colors to Madrid theme
\setbeamercolor{palette primary}{bg=mllavender3,fg=mlpurple}
\setbeamercolor{palette secondary}{bg=mllavender2,fg=mlpurple}
\setbeamercolor{palette tertiary}{bg=mllavender,fg=white}
\setbeamercolor{palette quaternary}{bg=mlpurple,fg=white}
\setbeamercolor{structure}{fg=mlpurple}
\setbeamercolor{frametitle}{fg=mlpurple,bg=mllavender3}
\setbeamercolor{block title}{bg=mllavender2,fg=mlpurple}
\setbeamercolor{block body}{bg=mllavender4,fg=black}

\setbeamertemplate{navigation symbols}{}
\setbeamertemplate{itemize items}[circle]
\setbeamersize{text margin left=5mm,text margin right=5mm}

% Bottom note command
\newcommand{\bottomnote}[1]{%
\vfill
\vspace{-2mm}
\textcolor{mllavender2}{\rule{\textwidth}{0.4pt}}
\vspace{1mm}
\footnotesize\textbf{#1}
}

\title{Agent Frameworks and Tools}
\subtitle{Week 6: LangGraph, AutoGen, and Production Patterns}
\author{Agentic Artificial Intelligence}
\institute{PhD Course}
\date{2025}

\begin{document}

% Copyright footer
\setbeamertemplate{footline}{
    \hbox{\begin{beamercolorbox}[wd=\paperwidth,ht=2.5ex,dp=1ex,center]{author in head/foot}
    \tiny (c) Joerg Osterrieder 2025
    \end{beamercolorbox}}
}

% ==================== SLIDE 1: Title ====================
\begin{frame}[plain]
\vspace{1.5cm}
\begin{center}
{\Huge\textcolor{mlpurple}{Agent Frameworks and Tools}}\\[0.5cm]
{\Large Week 6: LangGraph, AutoGen, and Production Patterns}\\[1.5cm]
{\normalsize PhD Course in Agentic Artificial Intelligence}\\[0.5cm]
{\small 12-Week Research-Level Course}
\end{center}
\end{frame}

% ==================== SLIDE 2: Learning Objectives ====================
\begin{frame}[t]{Learning Objectives}
\textbf{Bloom's Taxonomy Levels Covered}
\begin{itemize}
\item \textbf{Remember}: Define state machines (flow control), graphs, orchestration
\item \textbf{Understand}: Explain LangGraph vs LangChain architecture differences
\item \textbf{Apply}: Implement a stateful agent using LangGraph
\item \textbf{Analyze}: Compare framework trade-offs for different use cases
\item \textbf{Evaluate}: Assess production readiness of agent frameworks
\item \textbf{Create}: Design a custom agent orchestration system
\end{itemize}
\bottomnote{By end of lecture, you will understand how to select and use production agent frameworks.}
\end{frame}

% ==================== SLIDE 3: Framework Landscape ====================
\begin{frame}[t]{Agent Framework Landscape}
\textbf{First Generation (2023)}
\begin{itemize}
\item \textbf{LangChain}: Chain-based composition, extensive tool library
\item \textbf{LlamaIndex}: Data-centric, RAG-focused framework
\end{itemize}
\vspace{0.3cm}
\textbf{Second Generation (2024)}
\begin{itemize}
\item \textbf{LangGraph}: Graph-based state machines, cycles allowed
\item \textbf{AutoGen}: Multi-agent conversations, code execution
\item \textbf{CrewAI}: Role-based multi-agent orchestration
\end{itemize}
\vspace{0.3cm}
\textbf{Enterprise Frameworks}
\begin{itemize}
\item \textbf{Semantic Kernel} (Microsoft): Enterprise integration
\item \textbf{Haystack} (deepset): Production NLP pipelines
\end{itemize}
\bottomnote{Framework choice depends on: complexity, multi-agent needs, production requirements.}
\end{frame}

% ==================== SLIDE 4: Framework Comparison ====================
\begin{frame}[t]{Framework Comparison}
\begin{center}
\includegraphics[width=0.55\textwidth]{01_framework_comparison/framework_comparison.pdf}
\end{center}
\bottomnote{LangGraph excels at flexibility and state management; CrewAI at ease of use.}
\end{frame}

% ==================== SLIDE 5: LangGraph Architecture ====================
\begin{frame}[t]{LangGraph: Graph-Based Agents}
\textbf{Core Concepts}
\begin{itemize}
\item \textbf{StateGraph}: Define nodes and edges as a directed graph
\item \textbf{Nodes}: Functions that modify state
\item \textbf{Edges}: Conditional routing between nodes
\item \textbf{Cycles}: Enable iterative refinement loops
\end{itemize}
\vspace{0.3cm}
\textbf{Key Advantages}
\begin{itemize}
\item Explicit control flow visualization
\item Built-in checkpointing (state snapshots) and persistence
\item Human-in-the-loop interrupts
\item Streaming support for real-time output
\end{itemize}
\bottomnote{LangGraph treats agent logic as a stateful graph, not a linear chain.}
\end{frame}

% ==================== SLIDE 6: LangGraph Flow ====================
\begin{frame}[t]{LangGraph: State Machine Flow}
\begin{center}
\includegraphics[width=0.60\textwidth]{02_langgraph_flow/langgraph_flow.pdf}
\end{center}
\bottomnote{The graph cycles through Agent-Tools-Check until task completion.}
\end{frame}

% ==================== SLIDE 7: State Management ====================
\begin{frame}[t]{State Management in Agents}
\begin{center}
\includegraphics[width=0.55\textwidth]{03_state_management/state_management.pdf}
\end{center}
\bottomnote{Production agents typically require persistent state for reliability.}
\end{frame}

% ==================== SLIDE 8: LangGraph State Definition ====================
\begin{frame}[t]{Defining Agent State}
\textbf{TypedDict State Schema}
\begin{itemize}
\item Define state as Python TypedDict for type safety
\item Messages list tracks conversation history
\item Custom fields for domain-specific context
\end{itemize}
\vspace{0.3cm}
\textbf{State Updates}
\begin{itemize}
\item Nodes return partial state updates
\item Reducers merge updates (append, replace, custom)
\item Checkpointing persists state after each node
\end{itemize}
\vspace{0.3cm}
\textbf{Persistence Options}
\begin{itemize}
\item MemorySaver: In-memory (development)
\item SqliteSaver: SQLite database
\item PostgresSaver: PostgreSQL (production)
\end{itemize}
\bottomnote{State schema determines what information flows through the agent graph.}
\end{frame}

% ==================== SLIDE 9: AutoGen Overview ====================
\begin{frame}[t]{AutoGen: Multi-Agent Conversations}
\textbf{Core Abstraction}
\begin{itemize}
\item Agents communicate through message passing
\item ConversableAgent: Base class for all agents
\item UserProxyAgent: Represents human user
\item AssistantAgent: LLM-powered responder
\end{itemize}
\vspace{0.3cm}
\textbf{Key Features}
\begin{itemize}
\item \textbf{Code Execution}: Docker (containerized)/local code sandbox
\item \textbf{Group Chat}: Multi-agent discussions
\item \textbf{Nested Chats}: Hierarchical conversations
\item \textbf{Custom Replies}: Programmable response logic
\end{itemize}
\bottomnote{AutoGen excels at collaborative multi-agent problem solving.}
\end{frame}

% ==================== SLIDE 10: Orchestration Patterns ====================
\begin{frame}[t]{Orchestration Patterns}
\begin{center}
\includegraphics[width=0.60\textwidth]{04_orchestration_patterns/orchestration_patterns.pdf}
\end{center}
\bottomnote{Choose pattern based on task parallelizability and coordination needs.}
\end{frame}

% ==================== SLIDE 11: Production Considerations ====================
\begin{frame}[t]{Production Deployment}
\textbf{Reliability}
\begin{itemize}
\item Retry logic with exponential backoff
\item Graceful degradation on API failures
\item Timeout handling for long-running tasks
\end{itemize}
\vspace{0.3cm}
\textbf{Observability}
\begin{itemize}
\item LangSmith/LangFuse (trace visualization) for tracing
\item Structured logging with correlation IDs
\item Metrics: latency, token usage, success rate
\end{itemize}
\vspace{0.3cm}
\textbf{Security}
\begin{itemize}
\item Sandboxed code execution
\item Input validation and sanitization
\item Rate limiting and cost controls
\end{itemize}
\bottomnote{Production agents require robust error handling and monitoring.}
\end{frame}

% ==================== SLIDE 12: Framework Selection ====================
\begin{frame}[t]{Choosing a Framework}
\small
\begin{tabular}{lll}
\toprule
\textbf{Use Case} & \textbf{Recommended} & \textbf{Reason} \\
\midrule
Simple RAG chatbot & LangChain & Easy setup \\
Complex workflows & LangGraph & Graph control \\
Multi-agent debate & AutoGen & Conversation \\
Role-based teams & CrewAI & Agent roles \\
Enterprise apps & Semantic Kernel & .NET/Azure \\
Custom pipeline & Build own & Full control \\
\bottomrule
\end{tabular}
\vspace{0.4cm}

\textbf{Decision Criteria}
\begin{itemize}
\item Complexity of agent logic and state management
\item Need for multi-agent coordination
\item Production requirements (observability, persistence)
\end{itemize}
\bottomnote{Start simple (LangChain), migrate to LangGraph when you need cycles/state.}
\end{frame}

% ==================== SLIDE 13: Readings ====================
\begin{frame}[t]{Required Readings}
\textbf{This Week}
\begin{itemize}
\item LangGraph Documentation: \url{langchain-ai.github.io/langgraph}
\item Wu et al. (2023). ``AutoGen: Enabling Next-Gen LLM Applications.'' arXiv:2308.08155
\end{itemize}
\vspace{0.3cm}
\textbf{Supplementary}
\begin{itemize}
\item Wooldridge \& Jennings (1995). ``Intelligent Agents: Theory and Practice.'' \textit{Knowledge Engineering Review}.
\item CrewAI Documentation: \url{docs.crewai.com}
\item Qiao et al. (2024). ``TaskWeaver: A Code-First Agent Framework.'' arXiv:2311.17541
\end{itemize}
\bottomnote{Focus on LangGraph docs -- hands-on implementation is essential.}
\end{frame}

% ==================== SLIDE 14: Summary ====================
\begin{frame}[t]{Summary and Key Takeaways}
\textbf{Key Concepts}
\begin{itemize}
\item \textbf{LangGraph}: Graph-based state machines with cycles
\item \textbf{AutoGen}: Conversational multi-agent framework
\item \textbf{State}: TypedDict schema with persistence
\item \textbf{Patterns}: Sequential, parallel, router, hierarchical
\end{itemize}
\vspace{0.3cm}
\textbf{Production Checklist}
\begin{itemize}
\item Checkpointing enabled
\item Error handling and retries
\item Observability instrumentation
\end{itemize}
\vspace{0.3cm}
\textbf{Next Week}
\begin{itemize}
\item Advanced RAG: Self-RAG, CRAG, Agentic RAG
\end{itemize}
\bottomnote{Framework mastery = understanding state management + orchestration patterns.}
\end{frame}

\end{document}
