\documentclass[8pt,aspectratio=169]{beamer}
\usetheme{Madrid}
\usepackage{graphicx}
\usepackage{booktabs}
\usepackage{adjustbox}
\usepackage{amsmath}
\usepackage{amssymb}
\usepackage{hyperref}

% Color definitions
\definecolor{mlblue}{RGB}{0,102,204}
\definecolor{mlpurple}{RGB}{51,51,178}
\definecolor{mllavender}{RGB}{173,173,224}
\definecolor{mllavender2}{RGB}{193,193,232}
\definecolor{mllavender3}{RGB}{204,204,235}
\definecolor{mllavender4}{RGB}{214,214,239}
\definecolor{mlorange}{RGB}{255,127,14}
\definecolor{mlgreen}{RGB}{44,160,44}
\definecolor{mlred}{RGB}{214,39,40}
\definecolor{mlgray}{RGB}{127,127,127}

\setbeamercolor{palette primary}{bg=mllavender3,fg=mlpurple}
\setbeamercolor{palette secondary}{bg=mllavender2,fg=mlpurple}
\setbeamercolor{palette tertiary}{bg=mllavender,fg=white}
\setbeamercolor{palette quaternary}{bg=mlpurple,fg=white}
\setbeamercolor{structure}{fg=mlpurple}
\setbeamercolor{frametitle}{fg=mlpurple,bg=mllavender3}
\setbeamercolor{block title}{bg=mllavender2,fg=mlpurple}
\setbeamercolor{block body}{bg=mllavender4,fg=black}

\setbeamertemplate{navigation symbols}{}
\setbeamertemplate{itemize items}[circle]
\setbeamersize{text margin left=5mm,text margin right=5mm}

\newcommand{\bottomnote}[1]{%
\vfill
\vspace{-2mm}
\textcolor{mllavender2}{\rule{\textwidth}{0.4pt}}
\vspace{1mm}
\footnotesize\textbf{#1}
}

\title{Research Frontiers}
\subtitle{Week 12: Open Problems and Future Directions}
\author{Agentic Artificial Intelligence}
\institute{PhD Course}
\date{2025}

\begin{document}

\setbeamertemplate{footline}{
    \hbox{\begin{beamercolorbox}[wd=\paperwidth,ht=2.5ex,dp=1ex,center]{author in head/foot}
    \tiny (c) Joerg Osterrieder 2025
    \end{beamercolorbox}}
}

% ==================== SLIDE 1: Title ====================
\begin{frame}[plain]
\vspace{1.5cm}
\begin{center}
{\Huge\textcolor{mlpurple}{Research Frontiers}}\\[0.5cm]
{\Large Week 12: Open Problems and Future Directions}\\[1.5cm]
{\normalsize PhD Course in Agentic Artificial Intelligence}\\[0.5cm]
{\small 12-Week Research-Level Course}
\end{center}
\end{frame}

% ==================== SLIDE 2: Learning Objectives ====================
\begin{frame}[t]{Learning Objectives}
\textbf{Bloom's Taxonomy Levels Covered}
\begin{itemize}
\item \textbf{Remember}: Define embodied agent (physical/virtual world), generative agent (simulated personas), world model (environment simulation)
\item \textbf{Understand}: Explain key open research problems in agent AI
\item \textbf{Apply}: Identify research opportunities in specific domains
\item \textbf{Analyze}: Compare different approaches to agent safety and alignment
\item \textbf{Evaluate}: Assess feasibility and impact of proposed research directions
\item \textbf{Create}: Design a research proposal for advancing agent capabilities
\end{itemize}
\bottomnote{By end of lecture, you will understand the research frontier in agentic AI.}
\end{frame}

% ==================== SLIDE 3: Field Timeline ====================
\begin{frame}[t]{The Rapid Evolution of Agent AI}
\textbf{2022: Foundation}
\begin{itemize}
\item Chain-of-Thought prompting (Wei et al.)
\item InstructGPT and RLHF alignment (OpenAI)
\end{itemize}
\vspace{0.2cm}
\textbf{2023: Emergence}
\begin{itemize}
\item ReAct paradigm (Yao et al.)
\item Reflexion self-improvement (Shinn et al.)
\item Generative Agents simulation (Park et al.)
\end{itemize}
\vspace{0.2cm}
\textbf{2024: Production}
\begin{itemize}
\item Claude Computer Use, GitHub Copilot Workspace
\item GraphRAG, advanced RAG architectures
\item Multi-agent frameworks mature
\end{itemize}
\vspace{0.2cm}
\textbf{2025+: What's Next?}
\begin{itemize}
\item World models, embodied agents, long-horizon planning
\end{itemize}
\bottomnote{From reasoning (2022) to production systems (2024) in just two years.}
\end{frame}

% ==================== SLIDE 4: Research Timeline Chart ====================
\begin{frame}[t]{Agent Research Timeline}
\begin{center}
\includegraphics[width=0.65\textwidth]{01_research_timeline/research_timeline.pdf}
\end{center}
\bottomnote{From reasoning (2022) to production systems (2024) in just two years.}
\end{frame}

% ==================== SLIDE 5: Open Problems Overview ====================
\begin{frame}[t]{Key Open Research Problems}
\textbf{Capability Gaps}
\begin{itemize}
\item \textbf{Long-horizon planning}: Current agents struggle beyond 10-20 steps
\item \textbf{World modeling}: Learning accurate environment dynamics
\item \textbf{Compositional generalization}: Transfer to novel task combinations
\end{itemize}
\vspace{0.3cm}
\textbf{Safety and Alignment}
\begin{itemize}
\item \textbf{Scalable oversight}: How to supervise agents we can't fully understand?
\item \textbf{Goal stability}: Preventing goal drift during execution
\item \textbf{Corrigibility}: Ensuring agents remain controllable
\end{itemize}
\vspace{0.3cm}
\textbf{Infrastructure}
\begin{itemize}
\item \textbf{Evaluation}: Benchmarks that predict real-world performance
\item \textbf{Memory}: Efficient, scalable long-term memory systems
\end{itemize}
\bottomnote{These interconnected challenges define the research agenda.}
\end{frame}

% ==================== SLIDE 6: Open Problems Chart ====================
\begin{frame}[t]{Open Research Problems}
\begin{center}
\includegraphics[width=0.60\textwidth]{02_open_problems/open_problems.pdf}
\end{center}
\bottomnote{These interconnected challenges define the research agenda.}
\end{frame}

% ==================== SLIDE 7: Safety Challenges ====================
\begin{frame}[t]{Agent Safety Challenges}
\textbf{Alignment at Inference Time}
\begin{itemize}
\item Training-time alignment may not hold during multi-step execution
\item Agents can find loopholes in instructions (specification gaming)
\item Emergent behaviors from agent interactions
\end{itemize}
\vspace{0.3cm}
\textbf{Key Safety Research Areas}
\begin{itemize}
\item \textbf{Constitutional AI}: Principle-based self-supervision (Anthropic)
\item \textbf{Debate}: Agents argue, humans judge
\item \textbf{Interpretability}: Understanding agent reasoning
\item \textbf{Sandboxing}: Limiting agent action space
\end{itemize}
\vspace{0.3cm}
\textbf{Unresolved Questions}
\begin{itemize}
\item How do we align agents smarter than evaluators?
\item What governance structures for autonomous agents?
\end{itemize}
\bottomnote{Safety research must scale with capability improvements.}
\end{frame}

% ==================== SLIDE 8: Safety Challenges Chart ====================
\begin{frame}[t]{Safety Challenges}
\begin{center}
\includegraphics[width=0.60\textwidth]{03_safety_challenges/safety_challenges.pdf}
\end{center}
\bottomnote{Safety research is critical for responsible agent deployment.}
\end{frame}

% ==================== SLIDE 9: World Models and Embodied Agents ====================
\begin{frame}[t]{World Models and Embodied Agents}
\textbf{World Models}
\begin{itemize}
\item Learn internal representation of environment dynamics
\item Enable mental simulation before acting (``thinking ahead'')
\item Key challenge: Learning from limited interaction data
\end{itemize}
\vspace{0.3cm}
\textbf{Embodied Agents}
\begin{itemize}
\item Agents that interact with physical or simulated worlds
\item Examples: Robotics, game environments, simulations
\item \textbf{Voyager} (Wang et al.): Open-ended learning in Minecraft
\end{itemize}
\vspace{0.3cm}
\textbf{Research Directions}
\begin{itemize}
\item Sim-to-real transfer: Train in simulation, deploy in reality
\item Multimodal perception: Vision, audio, proprioception
\item Continuous learning: Adapt to changing environments
\end{itemize}
\bottomnote{World models enable agents to plan without trial-and-error.}
\end{frame}

% ==================== SLIDE 10: Generative Agents ====================
\begin{frame}[t]{Generative Agents: Simulated Societies}
\textbf{Generative Agents (Park et al., 2023)}
\begin{itemize}
\item Simulated personas in interactive environments
\item Agents maintain: identity, memories, plans, relationships
\item Emergent social behaviors: parties, information spread, coordination
\end{itemize}
\vspace{0.3cm}
\textbf{Key Architecture Components}
\begin{itemize}
\item \textbf{Memory stream}: Record of observations and reflections
\item \textbf{Retrieval}: Access relevant memories for decisions
\item \textbf{Reflection}: Synthesize higher-level insights
\item \textbf{Planning}: Daily schedules and goal pursuit
\end{itemize}
\vspace{0.3cm}
\textbf{Implications}
\begin{itemize}
\item Social science simulation at scale
\item Testing policies in simulated societies
\item Understanding emergent collective behavior
\end{itemize}
\bottomnote{Generative agents enable computational social science experiments.}
\end{frame}

% ==================== SLIDE 11: Future Directions ====================
\begin{frame}[t]{Future Directions}
\textbf{Near-Term (1-2 years)}
\begin{itemize}
\item More reliable multi-step execution
\item Better tool use and API integration
\item Production-ready multi-agent orchestration
\end{itemize}
\vspace{0.3cm}
\textbf{Medium-Term (3-5 years)}
\begin{itemize}
\item Agents with persistent, updateable world models
\item Effective long-term memory at scale
\item Robust sim-to-real transfer for embodied agents
\end{itemize}
\vspace{0.3cm}
\textbf{Long-Term (5+ years)}
\begin{itemize}
\item Agents that learn continuously from experience
\item Multi-agent societies with emergent specialization
\item General-purpose assistants for complex domains
\end{itemize}
\bottomnote{Progress requires interdisciplinary collaboration.}
\end{frame}

% ==================== SLIDE 12: Future Directions Chart ====================
\begin{frame}[t]{Future Directions}
\begin{center}
\includegraphics[width=0.65\textwidth]{04_future_directions/future_directions.pdf}
\end{center}
\bottomnote{Progress requires interdisciplinary collaboration.}
\end{frame}

% ==================== SLIDE 13: Key Papers ====================
\begin{frame}[t]{Required Readings}
\textbf{Foundational}
\begin{itemize}
\item Wang et al. (2023). ``Voyager: An Open-Ended Embodied Agent with LLMs.'' arXiv:2305.16291
\item Park et al. (2023). ``Generative Agents: Interactive Simulacra of Human Behavior.'' arXiv:2304.03442
\item Bai et al. (2022). ``Constitutional AI: Harmlessness from AI Feedback.'' arXiv:2212.08073
\end{itemize}
\vspace{0.3cm}
\textbf{Perspectives}
\begin{itemize}
\item Xi et al. (2023). ``The Rise and Potential of LLM Based Agents: A Survey.'' arXiv:2309.07864
\item Sumers et al. (2024). ``Language Agents: From Next-Token Prediction to Digital Automation.''
\end{itemize}
\bottomnote{These papers define the frontier of agent research.}
\end{frame}

% ==================== SLIDE 14: Course Summary ====================
\begin{frame}[t]{Course Summary: 12-Week Journey}
\textbf{Foundations (Weeks 1-2)}
\begin{itemize}
\item Agents, ReAct paradigm, LLM foundations, CoT/ToT prompting
\end{itemize}
\vspace{0.2cm}
\textbf{Capabilities (Weeks 3-5)}
\begin{itemize}
\item Tool use, MCP, planning, Reflexion, multi-agent architectures
\end{itemize}
\vspace{0.2cm}
\textbf{Frameworks (Week 6)}
\begin{itemize}
\item LangGraph, AutoGen, CrewAI, production patterns
\end{itemize}
\vspace{0.2cm}
\textbf{Knowledge (Weeks 7-9)}
\begin{itemize}
\item Advanced RAG, GraphRAG, hallucination prevention
\end{itemize}
\vspace{0.2cm}
\textbf{Applications (Weeks 10-12)}
\begin{itemize}
\item Evaluation, domain applications, research frontiers
\end{itemize}
\bottomnote{Agents = LLM + Memory + Tools + Planning + Evaluation}
\end{frame}

% ==================== SLIDE 15: Final Summary ====================
\begin{frame}[t]{Key Takeaways and Next Steps}
\textbf{Core Formula}
\begin{itemize}
\item Agent = LLM + Memory + Tools + Planning + Evaluation
\item Each component is an active research area
\end{itemize}
\vspace{0.3cm}
\textbf{Where to Focus Research}
\begin{itemize}
\item \textbf{High impact}: Long-horizon planning, safety, evaluation
\item \textbf{Underexplored}: Multi-agent emergence, world models
\item \textbf{Application-driven}: Domain-specific agent architectures
\end{itemize}
\vspace{0.3cm}
\textbf{Final Project Directions}
\begin{itemize}
\item Novel agent architecture for a specific domain
\item Improved evaluation methodology
\item Safety or alignment technique
\item Multi-agent coordination mechanism
\end{itemize}
\bottomnote{Thank you for participating in this course!}
\end{frame}

\end{document}
