\documentclass[8pt,aspectratio=169]{beamer}
\usetheme{Madrid}
\usepackage{graphicx}
\usepackage{booktabs}
\usepackage{amsmath}
\usepackage{amssymb}
\usepackage{hyperref}

\definecolor{mlblue}{RGB}{0,102,204}
\definecolor{mlpurple}{RGB}{51,51,178}
\definecolor{mllavender}{RGB}{173,173,224}
\definecolor{mllavender2}{RGB}{193,193,232}
\definecolor{mllavender3}{RGB}{204,204,235}
\definecolor{mllavender4}{RGB}{214,214,239}
\definecolor{mlorange}{RGB}{255,127,14}
\definecolor{mlgreen}{RGB}{44,160,44}
\definecolor{mlred}{RGB}{214,39,40}
\definecolor{mlgray}{RGB}{127,127,127}

\setbeamercolor{palette primary}{bg=mllavender3,fg=mlpurple}
\setbeamercolor{palette secondary}{bg=mllavender2,fg=mlpurple}
\setbeamercolor{palette tertiary}{bg=mllavender,fg=white}
\setbeamercolor{palette quaternary}{bg=mlpurple,fg=white}
\setbeamercolor{structure}{fg=mlpurple}
\setbeamercolor{frametitle}{fg=mlpurple,bg=mllavender3}

\setbeamertemplate{navigation symbols}{}
\setbeamertemplate{itemize items}[circle]
\setbeamersize{text margin left=5mm,text margin right=5mm}

\newcommand{\bottomnote}[1]{\vfill\vspace{-2mm}\textcolor{mllavender2}{\rule{\textwidth}{0.4pt}}\vspace{1mm}\footnotesize\textbf{#1}}

\title{Hallucination Prevention}
\subtitle{Week 9: Verification, Grounding, and Factuality}
\author{Agentic Artificial Intelligence}
\institute{PhD Course}
\date{2025}

\begin{document}

\setbeamertemplate{footline}{\hbox{\begin{beamercolorbox}[wd=\paperwidth,ht=2.5ex,dp=1ex,center]{author in head/foot}\tiny (c) Joerg Osterrieder 2025\end{beamercolorbox}}}

\begin{frame}[plain]
\vspace{1.5cm}
\begin{center}
{\Huge\textcolor{mlpurple}{Hallucination Prevention}}\\[0.5cm]
{\Large Week 9: Verification, Grounding, and Factuality}\\[1.5cm]
{\normalsize PhD Course in Agentic Artificial Intelligence}
\end{center}
\end{frame}

\begin{frame}[t]{Hallucination Types}
\begin{center}
\includegraphics[width=0.60\textwidth]{01_hallucination_types/hallucination_types.pdf}
\end{center}
\bottomnote{Factual hallucinations are highest risk; instruction drift (straying from task) is most common.}
\end{frame}

\begin{frame}[t]{Chain-of-Verification Pipeline}
\begin{center}
\includegraphics[width=0.65\textwidth]{02_verification_pipeline/verification_pipeline.pdf}
\end{center}
\bottomnote{Independent verification prevents confirmation bias (favoring initial beliefs).}
\end{frame}

\begin{frame}[t]{FActScore Evaluation}
\begin{center}
\includegraphics[width=0.60\textwidth]{03_factscore/factscore.pdf}
\end{center}
\bottomnote{FActScore measures atomic fact precision against knowledge sources.}
\end{frame}

\begin{frame}[t]{Mitigation Strategy Comparison}
\begin{center}
\includegraphics[width=0.60\textwidth]{04_mitigation_strategies/mitigation_strategies.pdf}
\end{center}
\bottomnote{Chain-of-Verification offers best accuracy/latency trade-off.}
\end{frame}

\begin{frame}[t]{Key Papers}
\textbf{This Week}
\begin{itemize}
\item Ji et al. (2023). ``Survey of Hallucination in NLG.'' arXiv:2202.03629
\item Min et al. (2023). ``FActScore: Fine-grained Atomic Evaluation.'' arXiv:2305.14251
\item Dhuliawala et al. (2023). ``Chain-of-Verification.'' arXiv:2309.11495
\end{itemize}
\bottomnote{Start with the hallucination survey for taxonomy and scope.}
\end{frame}

\begin{frame}[t]{Summary}
\textbf{Key Concepts}
\begin{itemize}
\item \textbf{Types}: Factual, faithfulness, instruction hallucinations
\item \textbf{Detection}: FActScore, claim decomposition (split into atomic facts), verification
\item \textbf{Prevention}: Grounding (anchor to sources), self-consistency, multi-agent review
\end{itemize}
\vspace{0.3cm}
\textbf{Next Week}
\begin{itemize}
\item Agent Evaluation and Benchmarking
\end{itemize}
\bottomnote{Prevention > Detection > Correction for production systems.}
\end{frame}

\end{document}
