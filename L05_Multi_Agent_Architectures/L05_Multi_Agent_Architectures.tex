\documentclass[8pt,aspectratio=169]{beamer}
\usetheme{Madrid}
\usepackage{graphicx}
\usepackage{booktabs}
\usepackage{amsmath}
\usepackage{amssymb}

\definecolor{mlblue}{RGB}{0,102,204}
\definecolor{mlpurple}{RGB}{51,51,178}
\definecolor{mllavender}{RGB}{173,173,224}
\definecolor{mllavender2}{RGB}{193,193,232}
\definecolor{mllavender3}{RGB}{204,204,235}
\definecolor{mlorange}{RGB}{255,127,14}
\definecolor{mlgreen}{RGB}{44,160,44}
\definecolor{mlgray}{RGB}{127,127,127}

\setbeamercolor{palette primary}{bg=mllavender3,fg=mlpurple}
\setbeamercolor{palette secondary}{bg=mllavender2,fg=mlpurple}
\setbeamercolor{palette tertiary}{bg=mllavender,fg=white}
\setbeamercolor{structure}{fg=mlpurple}
\setbeamercolor{frametitle}{fg=mlpurple,bg=mllavender3}

\setbeamertemplate{navigation symbols}{}
\setbeamertemplate{itemize items}[circle]
\setbeamersize{text margin left=5mm,text margin right=5mm}

\newcommand{\bottomnote}[1]{\vspace{\fill}\vspace{-2mm}\textcolor{mllavender2}{\rule{\textwidth}{0.4pt}}\vspace{1mm}{\footnotesize\textbf{#1}}}

\title{Multi-Agent Architectures}
\subtitle{Week 5: Collaborative AI Systems}
\author{Agentic Artificial Intelligence}
\date{2025}

\begin{document}

\setbeamertemplate{footline}{\hbox{\begin{beamercolorbox}[wd=\paperwidth,ht=2.5ex,dp=1ex,center]{author in head/foot}\tiny (c) Joerg Osterrieder 2025\end{beamercolorbox}}}

\begin{frame}[plain]
\vspace{1.5cm}
\begin{center}
{\Huge\textcolor{mlpurple}{Multi-Agent Architectures}}\\[0.5cm]
{\Large Week 5: Collaborative AI Systems}\\[1.5cm]
{\normalsize PhD Course in Agentic Artificial Intelligence}
\end{center}
\end{frame}

\begin{frame}[t]{Learning Objectives}
\textbf{Bloom's Taxonomy Levels}
\begin{itemize}
\item \textbf{Remember}: Define multi-agent systems, coordination, communication
\item \textbf{Understand}: Explain different agent topologies and roles
\item \textbf{Apply}: Implement message passing between agents
\item \textbf{Analyze}: Compare centralized vs. decentralized architectures
\item \textbf{Evaluate}: Assess trade-offs in multi-agent design
\item \textbf{Create}: Design a multi-agent system for a complex task
\end{itemize}
\bottomnote{Multi-agent systems enable complex tasks beyond single-agent capabilities.}
\end{frame}

\begin{frame}[t]{Why Multi-Agent Systems?}
\textbf{Single Agent Limitations}
\begin{itemize}
\item Context window constraints
\item No specialization (jack of all trades)
\item Sequential processing bottleneck
\end{itemize}
\vspace{0.3cm}
\textbf{Multi-Agent Benefits}
\begin{itemize}
\item Role specialization (expert agents)
\item Parallel task execution
\item Divide and conquer complex problems
\item Emergent collective intelligence
\end{itemize}
\bottomnote{Division of labor amplifies capabilities beyond individual agents.}
\end{frame}

\begin{frame}[t]{Communication Topologies}
\begin{center}
\includegraphics[width=0.70\textwidth]{01_communication_topology/communication_topology.pdf}
\end{center}
\bottomnote{Topology choice affects coordination overhead and flexibility.}
\end{frame}

\begin{frame}[t]{Agent Role Specialization}
\begin{center}
\includegraphics[width=0.60\textwidth]{02_role_specialization/role_specialization.pdf}
\end{center}
\bottomnote{Specialized roles enable division of cognitive labor.}
\end{frame}

\begin{frame}[t]{AutoGen Framework}
\textbf{Key Concepts} (Wu et al., 2023)
\begin{itemize}
\item Conversable agents with defined roles
\item Multi-turn message exchange
\item Human-in-the-loop (human oversight) integration
\end{itemize}
\vspace{0.3cm}
\textbf{Agent Types}
\begin{itemize}
\item \textbf{AssistantAgent}: LLM-powered responder
\item \textbf{UserProxyAgent}: Executes code, proxies human
\item \textbf{GroupChat}: Multi-agent conversation manager
\end{itemize}
\bottomnote{AutoGen simplifies multi-agent orchestration with conversation patterns.}
\end{frame}

\begin{frame}[t]{MetaGPT: Software Company Simulation}
\textbf{Key Idea} (Hong et al., 2023)
\begin{itemize}
\item Agents simulate software development roles
\item Product Manager, Architect, Engineer, QA
\item Structured output protocols between roles
\end{itemize}
\vspace{0.3cm}
\textbf{Communication Protocol}
\begin{itemize}
\item Standardized documents (PRD = Product Reqs, Design Doc)
\item Explicit handoff criteria between roles
\item Version control for artifacts
\end{itemize}
\bottomnote{Structured protocols reduce miscommunication in multi-agent systems.}
\end{frame}

\begin{frame}[t]{ChatDev: End-to-End Development}
\textbf{Key Idea} (Qian et al., 2024)
\begin{itemize}
\item Chat-based software development simulation
\item Phase-based workflow (Design, Coding, Testing)
\item Role-play dialogues between agents
\end{itemize}
\vspace{0.3cm}
\textbf{Phases}
\begin{enumerate}
\item \textbf{Design}: CEO + CTO discuss requirements
\item \textbf{Coding}: Programmer + Code Reviewer iterate
\item \textbf{Testing}: Tester + Programmer fix bugs
\item \textbf{Documentation}: Technical writer finalizes
\end{enumerate}
\bottomnote{ChatDev generates complete software from natural language specs.}
\end{frame}

\begin{frame}[t]{Coordination Mechanisms}
\textbf{Explicit Coordination}
\begin{itemize}
\item Central orchestrator assigns tasks
\item Defined protocols and interfaces
\item Clear responsibility boundaries
\end{itemize}
\vspace{0.3cm}
\textbf{Emergent Coordination}
\begin{itemize}
\item Agents negotiate in shared environment
\item Consensus through multi-round discussion
\item Voting or debate to resolve conflicts
\end{itemize}
\vspace{0.2cm}
\textbf{Hybrid Approaches}: Structure where needed, flexibility elsewhere
\bottomnote{Choose coordination mechanism based on task predictability.}
\end{frame}

\begin{frame}[t]{Message Passing Patterns}
\textbf{Request-Response}
\begin{itemize}
\item Agent A sends query to Agent B
\item B processes and returns result
\item Simple, synchronous, easy to trace
\end{itemize}
\vspace{0.3cm}
\textbf{Publish-Subscribe} (event broadcast)
\begin{itemize}
\item Agents subscribe to topics
\item Publishers broadcast to all subscribers
\item Decoupled, scalable, event-driven
\end{itemize}
\vspace{0.2cm}
\textbf{Shared Blackboard}
\begin{itemize}
\item Common workspace all agents can read/write
\item Good for collaborative problem-solving
\end{itemize}
\bottomnote{Pattern choice depends on coupling, latency, and complexity needs.}
\end{frame}

\begin{frame}[t]{Challenges in Multi-Agent Systems}
\textbf{Coordination Overhead}
\begin{itemize}
\item Communication cost increases with agent count
\item Synchronization delays
\item Context management across agents
\end{itemize}
\vspace{0.3cm}
\textbf{Failure Modes}
\begin{itemize}
\item Cascading errors (one agent fails, chain breaks)
\item Infinite loops (agents talking past each other)
\item Conflicting actions (race conditions)
\end{itemize}
\vspace{0.2cm}
\textbf{Mitigations}: Timeouts, circuit breakers (failure isolation), conflict resolution
\bottomnote{Multi-agent complexity requires robust error handling.}
\end{frame}

\begin{frame}[t]{Design Principles}
\textbf{Best Practices}
\begin{enumerate}
\item Start with minimal agents, add as needed
\item Define clear role boundaries
\item Use structured output formats
\item Implement graceful degradation
\item Monitor agent interactions
\end{enumerate}
\vspace{0.3cm}
\textbf{Anti-Patterns}
\begin{itemize}
\item Too many agents for simple tasks
\item Unclear responsibility boundaries
\item No termination conditions
\end{itemize}
\bottomnote{Simpler architectures are easier to debug and maintain.}
\end{frame}

\begin{frame}[t]{Required Readings}
\textbf{This Week}
\begin{itemize}
\item Tran et al. (2025). ``Multi-Agent Collaboration Survey.'' arXiv:2501.06322
\end{itemize}
\vspace{0.3cm}
\textbf{Supplementary}
\begin{itemize}
\item Wu et al. (2023). ``AutoGen.'' arXiv:2308.08155
\item Hong et al. (2023). ``MetaGPT.'' arXiv:2308.00352
\item Qian et al. (2024). ``ChatDev.'' arXiv:2307.07924
\end{itemize}
\bottomnote{Survey provides comprehensive overview; frameworks show implementation.}
\end{frame}

\begin{frame}[t]{Summary and Key Takeaways}
\textbf{Key Concepts}
\begin{itemize}
\item \textbf{Topologies}: Centralized, hierarchical, peer-to-peer
\item \textbf{Roles}: Orchestrator, planner, executor, critic
\item \textbf{Coordination}: Explicit protocols vs. emergent
\item \textbf{Frameworks}: AutoGen, MetaGPT, ChatDev
\end{itemize}
\vspace{0.3cm}
\textbf{Design Decisions}
\begin{itemize}
\item When to use multi-agent vs. single agent
\item Centralized vs. decentralized coordination
\item Structured vs. free-form communication
\end{itemize}
\vspace{0.2cm}
\textbf{Next Week}: Agent Frameworks (LangGraph, CrewAI)
\bottomnote{Multi-agent architectures enable complex collaborative intelligence.}
\end{frame}

\end{document}
